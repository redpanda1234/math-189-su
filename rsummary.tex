\documentclass[scipaper, 189]{pset}

\title{READING SUMMARY}
\author{Forest Kobayashi}
\affiliation{Department of Mathematics, Harvey Mudd College,
  Claremont, CA 91711}

\begin{document}
\begin{multicols}{2}
  \large
  Since the first paper I read turned out to be rather short, I decided
  to read a second and provide a summary for it as well.
  \section{Paper 1}
  For my summary, I chose to read a paper that might help in finding
  future directions for my project. I chose to examine
  \textit{Frequency-based Analysis of Financial Time Series}, written
  by Mohammad Hamed Izadi. In this paper, the author examines the
  spectral density of stock prices and log stock prices, and finds
  that their power spectral density is consistent with that predicted
  by a random walk model ($S(f) \propto 1/f^2$).

  First, the author gives a summary of previous work in the area,
  referencing many papers that have also found ($S(f) \propto 1/f^2$).
  Then, he performs his own analysis of some stock data, confirming
  the results by performing a fit in $\gamma, C_0$ on $S(f) = C_0 /
  f^\gamma$ for new datasets.

  The author then shows that for a random walk process
  \[
    X_t = \mu + X_{t-1} + \varepsilon_t
  \]
  (where $\mu$ is a drift parameter, and the $\varepsilon_i$ are all
  independent random variables), with some constraints, the power
  spectral density must be of the form $1/f^2$. The author then goes
  on to examine extracting correlation between low-frequency points in
  sequences of the form $1/f^2$, as such sequences are not memoryless.
  Thus, the author concludes, it might be possible to make some
  predictions about stock prices.
  \section{Paper 2}
  For my second paper, I chose to read \textit{Dynamics of the Dow
    Jones and the NASDAQ stock indexes}, by Fernando B.\ Duarte, J.\
  A.\ Tenreiro machado, and Gon\c{c}alo Monteiro Duarte. In this
  paper, the authors analyze the properties of financial data using
  techniques from dynamical systems. In particular, they use Takens'
  embedding theorem to determine whether a time series is ``a
  deterministic signal from a low-dimensional dynamical system.'' They
  state the theorem as follows:
  \begin{theorem}
    If a time series is one component of an attractor that can be
    represented by a smooth $d$-dimensional manifold (where $d \in
    \ZZ$), then the topological properties of the signal are
    equivalent to the topological properties of the embedding formed
    by the $m$-dimensional phase space vectors
    \[
      y(t) = \bk{s(t), s(t + \tau), s(t + 2\tau), \cdots, s(t +
        (m-1)\tau)}
    \]
    whenver $m > 2d + 1$. The vector $y(t)$ can be plotted in a
    $d$-dimensional space forming a curve in the Pseudo Phase Space.2
  \end{theorem}
  In the particular case of stock data, the authors use power
  functions to approximate the modulus of the fourier transform
  amplitudes, as a method for studying signal spectrums:
  \[
    \abs{\mc{F}\set{x(t)}} \cong p \omega^q \qquad p,q \in \RR
  \]
  then, the authors use a sliding-window Fourier Transform (to try and
  keep both time and frequency information), and then applied
  Pseudo Phase Plane analysis to calculate relationships between
  different time partitions for a stock.
\end{multicols}
\end{document}
